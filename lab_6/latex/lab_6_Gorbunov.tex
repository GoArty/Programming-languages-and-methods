\documentclass[a4paper, 14pt]{extarticle}

% Поля
%--------------------------------------
\usepackage{geometry}
\geometry{a4paper,tmargin=2cm,bmargin=2cm,lmargin=3cm,rmargin=1cm}
%--------------------------------------


%Russian-specific packages
%--------------------------------------
\usepackage[T2A]{fontenc}
\usepackage[utf8]{inputenc} 
\usepackage[english, main=russian]{babel}
%--------------------------------------

\usepackage{textcomp}

% Красная строка
%--------------------------------------
\usepackage{indentfirst}               
%--------------------------------------             


%Graphics
%--------------------------------------
\usepackage{graphicx}
\graphicspath{ {./images/} }
\usepackage{wrapfig}
%--------------------------------------

% Полуторный интервал
%--------------------------------------
\linespread{1.3}                    
%--------------------------------------

%Выравнивание и переносы
%--------------------------------------
% Избавляемся от переполнений
\sloppy
% Запрещаем разрыв страницы после первой строки абзаца
\clubpenalty=10000
% Запрещаем разрыв страницы после последней строки абзаца
\widowpenalty=10000
%--------------------------------------

%Списки
\usepackage{enumitem}

%Подписи
\usepackage{caption} 

%Гиперссылки
\usepackage{hyperref}

\hypersetup {
	unicode=true
}

%Рисунки
%--------------------------------------
\DeclareCaptionLabelSeparator*{emdash}{~--- }
\captionsetup[figure]{labelsep=emdash,font=onehalfspacing,position=bottom}
%--------------------------------------

\usepackage{tempora}

%Листинги
%--------------------------------------
\usepackage{listings}
\lstset{
  basicstyle=\ttfamily\footnotesize, 
  %basicstyle=\footnotesize\AnkaCoder,        % the size of the fonts that are used for the code
  breakatwhitespace=false,         % sets if automatic breaks shoulbd only happen at whitespace
  breaklines=true,                 % sets automatic line breaking
  captionpos=t,                    % sets the caption-position to bottom
  inputencoding=utf8,
  frame=single,                    % adds a frame around the code
  keepspaces=true,                 % keeps spaces in text, useful for keeping indentation of code (possibly needs columns=flexible)
  keywordstyle=\bf,       % keyword style
  numbers=left,                    % where to put the line-numbers; possible values are (none, left, right)
  numbersep=5pt,                   % how far the line-numbers are from the code
  xleftmargin=25pt,
  xrightmargin=25pt,
  showspaces=false,                % show spaces everywhere adding particular underscores; it overrides 'showstringspaces'
  showstringspaces=false,          % underline spaces within strings only
  showtabs=false,                  % show tabs within strings adding particular underscores
  stepnumber=1,                    % the step between two line-numbers. If it's 1, each line will be numbered
  tabsize=2,                       % sets default tabsize to 8 spaces
  title=\lstname                   % show the filename of files included with \lstinputlisting; also try caption instead of title
}
%--------------------------------------

%%% Математические пакеты %%%
%--------------------------------------
\usepackage{amsthm,amsfonts,amsmath,amssymb,amscd}  % Математические дополнения от AMS
\usepackage{mathtools}                              % Добавляет окружение multlined
\usepackage[perpage]{footmisc}
%--------------------------------------

%--------------------------------------
%			НАЧАЛО ДОКУМЕНТА
%--------------------------------------

\begin{document}

%--------------------------------------
%			ТИТУЛЬНЫЙ ЛИСТ
%--------------------------------------
\begin{titlepage}
\thispagestyle{empty}
\newpage


%Шапка титульного листа
%--------------------------------------
\vspace*{-60pt}
\hspace{-65pt}
\begin{minipage}{0.3\textwidth}
\hspace*{-20pt}\centering
\includegraphics[width=\textwidth]{emblem}
\end{minipage}
\begin{minipage}{0.67\textwidth}\small \textbf{
\vspace*{-0.7ex}
\hspace*{-6pt}\centerline{Министерство науки и высшего образования Российской Федерации}
\vspace*{-0.7ex}
\centerline{Федеральное государственное бюджетное образовательное учреждение }
\vspace*{-0.7ex}
\centerline{высшего образования}
\vspace*{-0.7ex}
\centerline{<<Московский государственный технический университет}
\vspace*{-0.7ex}
\centerline{имени Н.Э. Баумана}
\vspace*{-0.7ex}
\centerline{(национальный исследовательский университет)>>}
\vspace*{-0.7ex}
\centerline{(МГТУ им. Н.Э. Баумана)}}
\end{minipage}
%--------------------------------------

%Полосы
%--------------------------------------
\vspace{-25pt}
\hspace{-35pt}\rule{\textwidth}{2.3pt}

\vspace*{-20.3pt}
\hspace{-35pt}\rule{\textwidth}{0.4pt}
%--------------------------------------

\vspace{1.5ex}
\hspace{-35pt} \noindent \small ФАКУЛЬТЕТ\hspace{80pt} <<Информатика и системы управления>>

\vspace*{-16pt}
\hspace{47pt}\rule{0.83\textwidth}{0.4pt}

\vspace{0.5ex}
\hspace{-35pt} \noindent \small КАФЕДРА\hspace{50pt} <<Теоретическая информатика и компьютерные технологии>>

\vspace*{-16pt}
\hspace{30pt}\rule{0.866\textwidth}{0.4pt}
  
\vspace{11em}

\begin{center}
\Large {\bf Лабораторная работа № 6} \\ 
\large {\bf по курсу <<Языки и методы программирования>>} \\
\large <<Программа с графическим пользовательским интерфейсом>> 
\end{center}\normalsize

\vspace{8em}


\begin{flushright}
  {Студент группы ИУ9-21Б Горбунов А. Д. \hspace*{15pt}\\ 
  \vspace{2ex}
  Преподаватель Посевин Д. П.\hspace*{15pt}}
\end{flushright}

\bigskip

\vfill
 

\begin{center}
\textsl{Москва 2023}
\end{center}
\end{titlepage}
%--------------------------------------
%		КОНЕЦ ТИТУЛЬНОГО ЛИСТА
%--------------------------------------

\renewcommand{\ttdefault}{pcr}

\setlength{\tabcolsep}{3pt}
\newpage
\setcounter{page}{2}

\section{Задание}\label{Sect::task}
	Приобретение навыков разработки программ с графическим пользовательским интерфейсом на основе библиотеки swing.
 
    Прямоугольный треугольник выбранного цвета с катетами a и b , вписанный в окружность.
\section{Результаты}\label{Sect::res}

Исходный код программы представлен в листинге~\ref{lst:code1}, ~\ref{lst:code2}, ~\ref{lst:code3}, ~\ref{lst:code4}

\begin{figure}[!htb]
\begin{lstlisting}[language={},caption={класс PictureForm},label={lst:code1}]
import javax.swing.*;
import javax.swing.event.ChangeEvent;
import javax.swing.event.ChangeListener;
import java.awt.*;

public class PictureForm {
    private JPanel mainPanel;
    private JSpinner aSpinner;
    private CanvasPanel canvasPanel;
    private JSpinner bSpinner;
    private JSpinner color1Spinner;
    private JSpinner color2Spinner;
    private JSpinner color3Spinner;
    public PictureForm() {
        aSpinner.addChangeListener(new ChangeListener() {
            @Override
            public void stateChanged(ChangeEvent e) {
                int a = (int) aSpinner.getValue();
                canvasPanel.setA(a);
            }
        });
        bSpinner.addChangeListener(new ChangeListener() {
            @Override
            public void stateChanged(ChangeEvent e) {
                int b = (int) bSpinner.getValue();
                canvasPanel.setB(b);
            }
        });
        color1Spinner.addChangeListener(new ChangeListener() {
            @Override
            public void stateChanged(ChangeEvent e) {
                int b = (int) color1Spinner.getValue();
                canvasPanel.setColor1(b);
            }
        });
\end{lstlisting}
\end{figure}

\begin{figure}[!htb]
\begin{lstlisting}[language={},caption={класс PictureForm(продолжение)},label={lst:code2}]
        color2Spinner.addChangeListener(new ChangeListener() {
            @Override
            public void stateChanged(ChangeEvent e) {
                int b = (int) color2Spinner.getValue();
                canvasPanel.setColor2(b);
            }
        });
        color3Spinner.addChangeListener(new ChangeListener() {
            @Override
            public void stateChanged(ChangeEvent e) {
                int b = (int) color3Spinner.getValue();
                canvasPanel.setColor3(b);
            }
        });
        aSpinner.setValue(20);
        bSpinner.setValue(20);
        color1Spinner.setValue(0);
        color2Spinner.setValue(0);
        color1Spinner.setValue(0);
    }
    public static void main(String[] args) {
        JFrame frame = new JFrame("Right triangle inscribed in a circle");
        frame.setContentPane(new PictureForm().mainPanel);
        frame.setDefaultCloseOperation(JFrame.EXIT_ON_CLOSE);
        frame.pack();
        frame.setVisible(true);
    }
}
\end{lstlisting}
\end{figure}

\begin{figure}[!htb]
\begin{lstlisting}[language={},caption={класс CanvasPanel},label={lst:code3}]
import  javax.swing.*;
import java.awt.*;
public class CanvasPanel extends JPanel{
    private int a = 20;
    private int b = 20;
    private int color1 = 0;
    private int color2= 0;
    private int color3 = 0;
    protected void paintComponent(Graphics g){
        super.paintComponent(g);
        int xCenter = getWidth() / 2;
        int yCenter = getHeight() / 2;
        g.setColor(Color.getHSBColor((float)(color1*0.1),(float)(color2*0.1),(float)(color3*0.1)));
        g.drawPolygon(new int[] {xCenter, xCenter, xCenter+b}, new int[] {yCenter - a, yCenter, yCenter}, 3);
        g.fillPolygon(new int[] {xCenter, xCenter, xCenter+b}, new int[] {yCenter - a, yCenter, yCenter}, 3);
        int radius = (int) (Math.sqrt(a*a + b*b) / 2);
        g.setColor(Color.BLUE);
        g.drawOval((int)(xCenter+b/2-radius),(int)(yCenter-a/2-radius),radius*2,radius*2);
    }
\end{lstlisting}
\end{figure}

\begin{figure}[!htb]
\begin{lstlisting}[language={},caption={класс CanvasPanel(продолжение)},label={lst:code4}]
    protected void setA(int a){
        this.a = a;
        repaint();
    }
    protected void setB(int b){
        this.b = b;
        repaint();
    }
    protected void setColor1(int color){
        this.color1 = color;
        repaint();
    }
        protected void setColor2(int color){
        this.color2 = color;
        repaint();
    }
    protected void setColor3(int color){
        this.color3 = color;
        repaint();
    }
}
\end{lstlisting}
\end{figure}

\begin{figure}[!htb]
Результат запуска представлен на рисунке ~\ref{fig:picture_1.png}, ~\ref{fig:picture_2.png}, ~\ref{fig:picture_3.png}, ~\ref{fig:picture_4.png}
\end{figure}

\begin{figure}[!htb]
	\centering
	\includegraphics[width=0.8\textwidth]{picture_1.png}
\caption{Работа приложения(a==b)}
\label{fig:picture_1.png}
\end{figure}

\begin{figure}[!htb]
	\centering
	\includegraphics[width=0.8\textwidth]{picture_2.png}
\caption{Работа приложения(a>b)}
\label{fig:picture_2.png}
\end{figure}

\begin{figure}[!htb]
	\centering
	\includegraphics[width=0.8\textwidth]{picture_3.png}
\caption{Работа приложения(a<b)}
\label{fig:picture_3.png}
\end{figure}

\begin{figure}[!htb]
	\centering
	\includegraphics[width=0.8\textwidth]{picture_4.png}
\caption{Вид PictureForm.form}
\label{fig:picture_4.png}
\end{figure}

\end{document}

